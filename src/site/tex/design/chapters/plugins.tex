% When using TeXShop on the Mac, let it know the root document. The following must be one of the first 20 lines.
% !TEX root = ../design.tex

\chapter{CloudKeeper Extensibility}

% Abstract. What is the problem we want to solve?
CloudKeeper has a layered, modular architecture that allows user-defined module definitions, languages, and types, without any modifications to the CloudKeeper core.

\section{Requirements}

\subsection{Non-functional}

\begin{enumerate}
	\item \label{enum:NFR:P1} Extensibility: Allow users to extend CloudKeeper according to user needs, without having to ask CloudKeeper developers for support.
	\item \label{enum:NFR:P2} Usability: Extending CloudKeeper with user-defined modules, languages, or types should be easy (e.g., drag-and-drop-like solutions are preferable to editing configuration files).
	\item \label{enum:NFR:P3} Deployment: Easy (no downtime) deployment of functionality across different machines
	\item \label{enum:NFR:P4} Maintainability: Ease of development
\end{enumerate}

\subsection{Functional}

\begin{itemize}
	\item A repository of user-defined modules, languages, and types

		Reason: \ref{enum:NFR:P3}: Plug-Ins should be stored in a central location that is easily accessible to users.

	\item Embedding workflows in other workflows

		Reason: NFR \ref{enum:NFR:P1}, \ref{enum:NFR:P2}: The easiest way to implement new module definitions should be implementing it as a CloudKeeper workflow.

	\item Hot-swapping of plug-ins, i.e., add/remove plug-ins while the CloudKeeper environment is running

		Reason: NFR \ref{enum:NFR:P3}
\end{itemize}

\section{Module Definitions, Languages, and Types as Plug-Ins}

CloudKeeper is (will evolve into) a simple data-flow programming language, with which users develop workflows that link together modules implemented in arbitrary other programming languages (including CloudKeeper itself). As such, CloudKeeper by design is a universal tool for building new tools, and easy extensibility is a core requirement. There are many examples for similar requirements in other products: For instance, higher-level programming languages like Perl, Python, Java, etc.\ all support extensions written in C. Likewise, database systems such as PostgreSQL support user-defined functions, types, and languages as well.

In CloudKeeper, all module definitions, languages, and types are implemented as plug-ins. This gives the advantages of a modular and lean architecture with well-defined interfaces and a clear separation of concerns. All plug-ins have the following properties:
%
\begin{enumerate}
	\item Name: Each plug-in has a name, which follows the reverse-domain-name notation (like Java classes). It is advisable (but not enforced) that different types of plug-ins are kept in different namespaces. E.g., standard CloudKeeper modules reside in the namespace \texttt{com.svbio.cloudkeeper.modules}, whereas language definitions are kept in \texttt{com.svbio.cloudkeeper.languages}.

	\item Macros: Each plug-in can provide a macro generator that will be invoked before the execution phase. These macros can be used in the configuration of other plugins, or from the module code itself. As an example, a bash language plugin may define the macro \texttt{language.bash.exe} that contains the path to the bash executable on the current system. The module declaration for a module ``XYZ'' implemented in bash may then define its command-line invocation string as \texttt{"\$\{language.bash.exe\}" "\$\{modules.current.path\}/xyz.sh"}, where \texttt{modules.current.path} is another macro (provided by the CloudKeeper core).
\end{enumerate}

\subsection{Module Declarations}

Declarations of modules differ for simple modules and composite modules: Since a simple module is implemented in an arbitrary language, crucial information about in-ports and out-ports cannot be inferred but needs to be explicitly declared.

\paragraph{Simple-Module Declarations}

A simple module declaration contains everything any plug-in declaration contains, plus:
%
\begin{enumerate}
	\item In-ports: A list of the in-ports that this module expects.
	\item Out-ports: A list of the out-ports that this module provides.
	\item Command: The command line that should be used to invoke this module. This needs to be a single bash command. The execution happens as follows: First, all macros (i.e., substrings of form \texttt{\$\{macro-name\}}) are substituted with their actual values. Then, the actual command line is passed to bash for execution. Typical command lines might look like:
		\begin{itemize}
			\item \texttt{"\$\{language.bash.exe\}" "\$\{modules.current.path\}/xyz.sh"}
			\item \texttt{"\$\{language.java.exe\}" -jar "\$\{modules.current.path\}/xyz.jar" xzy}
		\end{itemize}
	\item Interface version: The interface version this module expects. This includes the low-level interface (e.g., the local directories where a module can expect its input, and where it should write its outputs to) as well as the language API.

		Should future version of CloudKeeper necessitate interface changes, CloudKeeoper can still execute versioned modules by using the appropriate legacy executor and legacy language API.
	\item Deterministic: Does this module return the same result whenever it is called with the same arguments? Modules that access information outside the control of CloudKeeper (e.g., pseudo-random number generators, the Internet, etc.) are \emph{not} deterministic.
\end{enumerate}

\paragraph{Composite Module Declarations}

A composite-module declaration contains everything any plug-in declaration contains, plus:
%
\begin{enumerate}
	\item Definition: A function that instantiates a \texttt{CompositeModule} object using the data structures as described in Section~\ref{sec:Workflow:ClassStructure}. This indirection exists because composite modules may be defined in different ways:
		\begin{itemize}
			\item Programmatically in Java, using the API to build the data structures, as described in Section~\ref{sec:Workflow:ClassStructure}. In this case, the definition is just a ``function pointer'' to the function containing the programmatic workflow definition.
			\item Using a domain-specific language as described in Chapter~\ref{ch:DSL}. In this case, the definition is the CloudKeeper function that translates the DSL code into the internal workflow representation.
		\end{itemize}
\end{enumerate}

\subsection{Language}

A language declaration currently contains only whatever a plugin declaration contains. In the future, language declarations may provide additional information. Possible options:
\begin{itemize}
	\item Whether a language is considered ``secure'' (execution can be sandboxed) or ``insecure'' (execution may theoretically bring down the whole machine)
	\item A validator that checks whether a module may properly execute (e.g., it may perform syntactical correctness checks of scripts during workflow definition time, rather than at runtime)
\end{itemize}

\subsection{Persistence of Plug-Ins}

Plug-Ins can be added and removed while CloudKeeper is running. For ease of deployment, CloudKeeper plug-ins are stored in the CloudKeeper distributed file store, under the \texttt{/system} hierarchy (see Chapter~\ref{ch:FileStore}). The default replication level for each file under in this directory is ``everywhere''.
Plug-ins are declared using JSON configuration files.

\begin{example}
Suppose we implement a module \texttt{com.svbio.cloudkeeper.modules.sum} in bash that sums up two integers. It consists of the following two files under \\\texttt{/system/com/svbio/cloudkeeper/modules/sum}:
\begin{itemize}
	\item \texttt{Info.json}: The module declaration. It might look as follows:
		\begin{lstlisting}[gobble=12]
			{
			   "type": "com.svbio.cloudkeeper.metadata.SimpleModuleDeclaration"
			   "name": "com.svbio.cloudkeeper.modules.sum",
			   "command": "${language.bash.exe}" ${modules.current.path}/sum.sh",
			   "inports": [ {
			           "name": "Num1",
			           "elementtype": "com.svbio.cloudkeeper.types.integer",
			           "dimension": 0
			       }, {
			           "name": "Num2",
			           "elementtype": "com.svbio.cloudkeeper.types.integer",
			           "dimension": 0
			       }
			   ],
			   "outports": [ {
			           "name": "Sum",
			           "elementtype": "com.svbio.cloudkeeper.types.integer",
			           "dimension": 0
			       }
			   ],
			   "deterministic": true
			}
		\end{lstlisting}
	\item \texttt{sum.sh}: The bash script that does the actual work. It might look as follows:
		\begin{lstlisting}[gobble=12]
			#!/bin/bash
			NUM1=$(getInput Num1)
			NUM2=$(getInput Num2)
			RESULT=$(echo "$NUM1 + $NUM2" | bc)
			setOutput Sum "$RESULT"
		\end{lstlisting}
\end{itemize}
Of course, languages and types (e.g., \texttt{com.svbio.cloudkeeper.types.integer}) are declared correspondingly.
\end{example}

\subsection{Class Structure}

Plugin declarations are represented by \texttt{PluginDefinition} objects. The class structure is shown in Figure~\ref{fig:PlugIns:Classes}.

Moreover, there is a \texttt{Repository} interface for loading plug-ins from persistent storage into \texttt{PluginDefinition} objects (see Figure~\ref{fig:PlugIns:RepositoryClasses}). An implementation complying with the previous section would scan the \texttt{/system} hierarchy in the file store, parse the \texttt{Info.json} files, and then instantiate the in-memory objects accordingly. For testing and during development, mock \texttt{Repository} objects can be used that generate \texttt{PluginDefinition} objects on-the-fly. When referring to plugins, it is generally cumbersomes having to deal with long names like \texttt{com.svbio.cloudkeeper.modules.sum}. Instead, it is convenient to have the notion of a current namespace and a set of imported names (like, e.g., in Java, where the current namespace is the package where the current class is defined, and names can be imported using the \texttt{import} statement). The equivalent in CloudKeeper is represented by the \texttt{Context} interface. A \texttt{Context} an be created from a \texttt{Repository} using the \texttt{createContextBuilder()} method, which sets the current namespace. Afterwards, the \texttt{addImport()} method may be used on the \texttt{ContextBuilder} object to bring other names into the current scope (context).

\begin{figure}
\tikzumlset{font=\ttfamily\small}
\centering
\begin{tikzpicture}
	\umlclass[x=0,y=6]{PortDefinition}{%
		+ name: String
	}{%
	}

	\umlclass[x=0,y=3,type=interface]{PortType}{%
		+ elementType: Class<?> \\
		+ dimensions: int
	}{%
	}

	\umlclass[x=6,y=15,type=interface]{PluginDefinition}{%
	}{%
		+ getName(): String\\
		+ getMacros(): Map<String, String>
	}

	\umlclass[x=3,y=12]{AbstractModuleDefinition}{%
		+ name: String
	}{%
		+ newInstance(name: String): AbstractModule
	}
	\umlinherit{AbstractModuleDefinition}{PluginDefinition}

	\umlclass[x=10,y=12]{LanguageDefinition}{%
	}{%
	}
	\umlinherit{LanguageDefinition}{PluginDefinition}

	\umlclass[x=0,y=9]{SimpleModuleDefinition}{%
		+ interfaceVersion: int \\
		+ command: String \\

	}{%
	}
	\umlinherit{SimpleModuleDefinition}{AbstractModuleDefinition}

	\umlclass[x=6,y=9]{CompositeModuleDefinition}{%
		+ definition:\\
		  (CompositeModule -> void)
	}{%
	}
	\umlinherit{CompositeModuleDefinition}{AbstractModuleDefinition}

	\umluniassoc[arg={+ type}]{PortDefinition}{PortType}
	\umluniaggreg[arg={+ inPorts}, mult=*, pos=0.5, anchor1=-140, anchor2=150]{SimpleModuleDefinition}{PortDefinition}
	\umluniaggreg[arg={+ outPorts}, mult=*, pos=0.5, anchor1=-40, anchor2=30]{SimpleModuleDefinition}{PortDefinition}
\end{tikzpicture}
\caption{Extensibiliy classes.\label{fig:PlugIns:Classes}}
\end{figure}

\begin{figure}
\tikzumlset{font=\ttfamily\small}
\centering
\begin{tikzpicture}
	\umlclass[x=0,y=0,type=interface]{Repository.ContextBuilder}{%
	}{%
		+ addImport(String name): \\ContextBuilder\\
		+ build(): Context
	}

	\umlclass[x=9,y=5,type=interface]{Context}{%
	}{%
		+ getPlugin(): PluginDefinition\\
		+ getModuleDefinition(): \\AbstractModuleDefinition\\
		+ getLanguageDefinition(): \\LanguageDefinition
	}

	\umlclass[x=9,y=0,type=interface]{Repository}{%
	}{%
		+ createContextBuilder\\(currentNamespace: String): \\ContextBuilder\\
		+ getMacros(): Map<String,String>
	}
	\umlinherit{Repository}{Context}
\end{tikzpicture}
\caption{Repository and Namespace Interfaces.\label{fig:PlugIns:RepositoryClasses}}
\end{figure}
